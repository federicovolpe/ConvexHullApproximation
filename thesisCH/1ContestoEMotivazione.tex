\chapter{Contesto e Motivazione}

\section{Contesto e Rilevanza del Problema}

Il problema trattato in questo documento si colloca nel campo della geometria computazionale, 
dove, in uno spazio a $D$ dimensioni, viene definita una regione ammissibile tramite un 
numero finito di iperpiani $E$, nota come \textit{Convex Hull}.
A causa del numero elevato di iperpiani e dell'alta dimensionalità che possono caratterizzare
questa regione, la manipolazione di questa può diventare particolarmente complessa 
e con tempi di elaborazione non indifferenti.
Risulta perciò necessario trovare un modo per poter rendere meno onerosi questi calcoli.\\

L'elaborato in questione si occupa di fornire metodi di approssimazione di gusci
convessi in \textit{D} dimensioni, le euristiche proposte devono soddisfare le seguenti condizioni:
\begin{itemize}
    \item Il politopo di approssimazione deve essere definito da un numero 
    di iperpiani (budget) fornito a priori $h$ per limitare la complessità della rappresentazione.
    \item Il politopo di approssimazione deve includere completamente la regione 
    definita dal guscio convesso originale.
    \item Quando il budget di iperpiani \textit{h} coincide con il numero di iperpiani del
    politopo di partenza \textit{E}, l'approssimazione e il politopo devono coincidere.
\end{itemize}

Tuttavia, l'alta dimensionalità e la complessità dei calcoli volumetrici rappresentano 
sfide significative che richiedono algoritmi efficienti per garantire una buona 
approssimazione in tempi ragionevoli.

\pagebreak

\section{Definizione del Guscio Convesso}

Il guscio convesso, o \textit{convex hull}, può essere definito in diversi modi equivalenti.\\
Una definizione rigorosa è fornita nel capitolo 2.2.4 di "Convex Optimization" 
\cite*{PolyhedraDefinition}, dove si afferma:


\begin{quote}
"Il guscio convesso è definito come l'insieme dei punti che soddisfano un numero 
finito di uguaglianze e disuguaglianze lineari secondo la seguente formula:

\[
\mathcal{P} = \left\{ x \; \middle| \; a_j^T x \leq b_j, \; j = 1, \ldots, m, \; c_j^T x = d_j, \; j = 1, \ldots, p \right\}.
\]
dove:
\begin{itemize}
    \item \( a_j \) vettore dei coefficienti delle disuguaglianze,
    \item \( b_j \) termine costante delle disuguaglianze,
    \item \( m \) numero totale di disuguaglianze,
    \item \( c_j \) vettore dei coefficienti delle uguaglianze,
    \item \( d_j \) termine noto delle uguaglianze,
    \item \( p \) numero totale di uguaglianze.
\end{itemize}
Un poliedro è quindi l'intersezione di un numero finito di semispazi \\
e iperpiani."
\end{quote}

Pertanto, è giusto notare che i poliedri possono essere chiusi,
aperti o anche vuoti in base alla loro definizione e alle condizioni 
imposte dai vincoli lineari. In questo caso ci si occuperà dell'approssimazione
di gusci convessi che siano chiusi e limitati.


\section{Approcci alla Risoluzione del Problema}

Diversi approcci per risolvere il problema dell'approssimazione del politopo:

\begin{itemize}
    \item \textbf{Approccio di Ottimizzazione}: 
    Tramite l'utilizzo di solutori si risolve un problema di Minimizzazione dell'area 
    inammissibile inclusa nel poliedro di approssimazione.
    \item \textbf{Approccio Geometrico}: 
    Vengono sfruttati differenti metodi geometrici per poter generare un 
    poliedro che meglio rappresenta quello di partenza.
\end{itemize}

Siccome gli algoritmi che si vogliono generare devono essere quanto 
piu prevedibili e di tempi di esecuzione noti, in questo documento si è scelto 
di effettuare uno studio comparativo di algoritmi che adottano approcci Geometrici.
Verranno dapprima ipotizzati differenti algoritmi per l'approssimazione 
di poliedri in 2 dimensioni, le cui performance verranno analizzate e paragonate successivamente.
Fra questi algoritmi, verranno selezionati quelli più promettenti per un'implementazione 
su poliedri in dimensioni più elevate tramite l'aiuto di software appositi. 

\section{possibili campi di applicazioni del progetto}
Una delle applicazioni più rilevanti del progetto è legata al concetto di collisione, 
ampiamente utilizzato nel campo della robotica. Un'approssimazione per eccesso di un 
corpo può rivelarsi fondamentale per prevenire contatti involontari durante le 
manovre dei robot, migliorando così la loro sicurezza e l'efficienza operativa. 
Inoltre, tale approccio può accelerare i calcoli necessari per determinare gli spazi 
di manovra, consentendo di spostarsi in ambienti complessi in modo più rapido 
e preciso.\\

L'approssimazione di un politopo n-dimensionale offre anche interessanti prospettive 
nella risoluzione di problemi di ottimizzazione. 
I gusci convessi sono frequentemente utilizzati in vari algoritmi di ottimizzazione, 
e in alcune situazioni, una soluzione subottimale derivante da un'approssimazione 
del guscio può essere considerata accettabile, specialmente in contesti dove la 
velocità di calcolo è cruciale.\\

Nel campo della computer vision, il progetto ha il potenziale di portare significativi 
avanzamenti nel riconoscimento di forme complesse e nel tracciamento di oggetti. 
La capacità di rappresentare oggetti tridimensionali come poliedri convessi potrebbe 
facilitare l'identificazione e il monitoraggio di oggetti in movimento, migliorando 
l'interazione tra i sistemi visivi e gli ambienti circostanti.\\

Nei problemi di clustering, approssimare i dati come un poliedro convesso può essere 
utile per identificare la forma e la struttura dei cluster. 
Ad esempio, i cluster di dati in spazi ad alta dimensione potrebbero essere approssimati 
come poliedri convessi, il che permette una migliore comprensione delle loro 
caratteristiche geometriche.\\

\section{Motivazione della ricerca} 
Il problema del calcolo del guscio convesso è stato ampiamente trattato in letteratura, 
con diverse soluzioni proposte, tra cui QuickHull\cite{QuickHull} e 
JarvisMarch\cite{JarvisMarch}. 
Tuttavia, ciò che distingue questo studio è l'obiettivo di sviluppare un metodo 
in grado di fornire un'approssimazione per eccesso del guscio convesso, 
con un numero di iperpiani fissato a priori.

In letteratura, esistono già studi relativi all'approssimazione del guscio convesso, 
che offrono soluzioni piuttosto precise, come descritto nel lavoro di riferimento 
\cite{CHApproximation}. 
Tuttavia, tali soluzioni non soddisfano pienamente la condizione fondamentale 
di inclusione del guscio convesso originario (condizione n.1) senza l'uso di 
adattamenti post-processing.

Gli algoritmi proposti in questo studio si distinguono proprio per l'originalità 
con cui affrontano il problema, offrendo una soluzione che rispetta la condizione 
di inclusione senza la necessità di aggiustamenti successivi. 
Questo rende il lavoro presentato non solo innovativo, ma anche potenzialmente 
più efficiente nel contesto specifico dell'approssimazione di gusci convessi.