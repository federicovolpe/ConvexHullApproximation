\chapter{Elevazione a n-dimensioni}

\section{Passi logici}
Date le conclusioni precedenti, si è deciso di procedere con lo sviluppo n-dimensionale
delgi algoritmi CuttingEdge e BoxCutting.
Poiché entrambi prevedono il calcolo delle aree, in più dimensioni richiederanno il 
calcolo di volumi n-dimensionali. 
Questi, come è noto, con l'aumentare delle dimensioni, avranno un grado di complessità
di calcolo sempre maggiore. Sono perciò previsti tempi di esecuzione gradualmente maggiori.

L'elevazione dell'algoritmo da semplice codice Java richiede l'utilizzo di strumenti 
avanzati, come Polymake, che consente di manipolare i poliedri con maggiore facilità, 
e del linguaggio Perl, con cui il software Polymake mostra una forte integrazione.

\subsection{L'algoritmo CuttingEdge n-dimensionale}
L'algoritmo CuttingEdge viene applicato a un poliedro di dimensione arbitraria 
con un budget prestabilito di iperpiani disponibili $h$. 
Il politopo iniziale utilizzato per l'approssimazione coincide con il poliedro originale.

Il processo per ogni faccia del poliedro si articola nei seguenti passi:

\begin{enumerate} 
    
    \item Individuazione dei coefficienti della faccia.
    
    \item Costruzione del politopo dato dal prolungamento degli iperpiani adiacenti
    a questa (inversione dei coefficienti della faccia);

    \item Individuazione del volume del politopo costruito;
    
    \item Selezione del'iperpiano producente il politopo con il volume inferiore fra
    tutti;

\end{enumerate}

Una volta individuato l'iperpiano migliore, questo viene rimosso dal politopo di approssimazione.

\subsection{L'algoritmo BoxCutting n-dimensionale}
Dato un poliedro in una dimensione arbitraria e $h$ rappresentante
il budget di iperpiani disponibili, 
viene costruito il parallelepipedo (Bounding Box) n dimensionale che lo racchiude.\\

L'algoritmo poi per ciascuna faccia procede come segue:

\begin{enumerate}
    
    \item Individuazione dei coefficienti della faccia;

    \item Costruzione del politopo incluso fra la faccia e la Bounding Box;
    
    \item Calcolo del volume del suddetto politopo;

    \item Selezione dell'iperpiano con il rispettivo volume maggiore;

\end{enumerate}

Una volta determinato l'iperpiano scelto, viene rimosso dalla bounding box il corrispettivo
volume compreso. L'insieme di iperpiani scelti determinerà il poliedro di approssimazione.

\section{Il Codice}
Per il calcolo del volume di un poliedro n-dimensionale esistono differenti metodi illustrati 
in questo documento: \cite{nDvolume} in cui è stata trovata una formula specifica.

Grazie all'utilizzo del software Polymake molte delle operazioni computazionalmente
rilevanti, come l'inversione dei semispazi o la rimozione/aggiunta di 
iperpiani ad un politopo, possono essere facilmente integrate nel codice.\\
Per la futura analisi è di importanza rilevante anche considerare che l'algoritmo 
CuttingEdge è stato dotato di una mappa per risparmiare operazioni di 
calcolo dei volumi ripetitive.

\subsection*{I politopi creati}

Esempio visivo in diverse angolazioni di uno step di approssimazione di un poliedro 
in 3 dimensioni.

\begin{figure}[H]
    \centering
    
    \newcounter{immagine} % Definizione del contatore
    \forloop{immagine}{1}{\value{immagine} < 5}{%
        \includegraphics[width=0.2\textwidth]{media/3Dproof/CuttingEdge/immagine\theimmagine.png} % Uso di \theimmagine per il numero del contatore
    }
    
    \caption{Esempio di approssimazione iniziale dell'algoritmo CuttingEdge}
\end{figure}

\begin{figure}[H]
    \centering
    
    \newcounter{img} % Definizione del contatore
    \forloop{img}{1}{\value{img} < 5}{%
        \includegraphics[width=0.2\textwidth]{media/3Dproof/BoxCutting/immagine\theimg.png} % Uso di \theimmagine per il numero del contatore
    }
    
    \caption{Esempio di approssimazione iniziale dell'algoritmo BoxCutting}
\end{figure}